\section{Why Models?}
\label{sec:model-abstract}

\projName\ is based on the observation that sensor data can be stored and processed much more efficiently by operating not on the actual measurements but instead on a model of the underlying reality that can be inferred from the measurements. What is a model? In statistics a model is a statement about reality, which could be in general of any form. For instance, a model could be a fundamental law (e.g., Newton's law of acceleration, stating that the force vector $\vec{F}$ can be inferred from the acceleration vector $\vec{a}$ and the mass $m$ through the formula $\vec{F}=m \vec{a}$) or a statement about a physical quantity in space and/or time in the past (e.g., the temperature at the entrance of the computer science department at UC San Diego on August 19, 2014 at 9am was between 68 and 70 degrees). But it could also be a predictive claim about the future (e.g., the average temperature of the earth will increase by at least 5 degrees between 2014 and 2050), or a  theory (e.g., a medical theory stating that the statins decrease the chance of a heart attack by more than 20\%).

Since we are interested in the processing of sensor data with a spatiotemporal component, in this work we restrict our focus to \emph{quantitative spatiotemporal models}; i.e., models that provide the value of a physical quantity for points in space and/or time. An example of a quantitative spatiotemporal model is a function outputting the temperature at the entrance of the computer science department at different points in time. Signal processing has long recognized the value of operating on models inferred from the measurements, instead of operating on the actual measurements themselves.\\

{\bf Advantages of models.} The reasons for preferring models over the raw measurements are multifold: Compared to raw sensor readings, models offer several advantages:

\begin{itemize}
\item \emph{They are functions with possibly infinite domains.} Raw sensor readings are merely discrete samples of an underlying continuous phenomenon (i.e., they provide values of the measured quantity only for a finite number of points in space and/or time). In contrast, models provide also all intermediate values, intuitively ``filling in" the gaps left by the raw measurements in the spatiotemporal dimensions. Having the values of the measured quantity for all points in space and/or time is especially important when joining two spatiotemporal signals on their spatiotemporal component, as otherwise the two signals may not be aligned. For instance, consider the following two datasets: A dataset containing air quality measurements taken at various locations and times at UC San Diego and another dataset containing GPS readings representing the location of a person walking around campus. If we want to compute the quality of the air the particular person was breathing during the walk, we would have to join these two datasets. However a conventional relational join on the space/time attributes of the two datasets will most probably yield the empty result, as the person may have never been at the exact time and location the air quality measurements were taken. Abstracting out each of these two sets of measurements through a corresponding model solves this problem and facilitates the join, as each model will provide the values for every point in space and time.
\item \emph{They offer predictive abilities.} In addition to providing values for points in space and/or time between those for which there exist raw measurements, models may also provide predictions for future points in time, thus allowing users to ask predictive queries. For instance, an air quality model may support queries about the expected air quality tomorrow. Intuitively, the predictive nature of the models stems from the fact that instead of focusing on the actual measurements, models instead capture the (typically recurring) pattern of the underlying phenomenon.
\item \emph{They improve accuracy.} By capturing the pattern of the underlying phenomenon, models may also be able to separate the noise (inevitably introduced in raw measurements due to the limited accuracy of sensors and other random factors) from the actual signal, leading to values that are more representative of the actual reality than the sensor measurements themselves. We will discuss in Section \ref{sec:compression} how models may separate the noise from the signal returning values that are more accurate than the original raw measurements.
\item \emph{They capture uncertainty information.} Even if a model cannot completely separate the noise from the signal, it can explicitly capture the uncertainty that exists in the reported value for the measured quantity. This uncertain information is then leveraged by the query processing algorithms to generate query answers that themselves capture uncertainty. We will outline in Section \ref{sec:probabilistic-models} different ways in which a model can capture uncertainty and describe their relationship to existing works in uncertain and probabilistic databases.
\item \emph{They can be represented compactly.} Finally, models can be most of the time represented more compactly than raw measurements. This not only reduces the storage requirements for the - typically large - sensor datasets, but leads in many cases also to more efficient query execution, as queries can often be evaluated directly on the compressed model representation, as we will discuss in Section \ref{sec:query-processing}.
\end{itemize}

We next describe how \projName\ incorporates models and their associated advantages into a relational DBMS.

